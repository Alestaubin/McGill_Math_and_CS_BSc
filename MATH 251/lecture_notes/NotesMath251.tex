\documentclass[12pt]{article}
\usepackage[paper=letterpaper,margin=2cm]{geometry}
\usepackage{amsmath}
\usepackage{amsthm} %needed for the proofs 
\usepackage{amssymb}
\usepackage{amsfonts}
\usepackage{newtxtext, newtxmath}
\usepackage{enumitem}
\usepackage{titling}
\usepackage{titlesec}
\usepackage[english]{babel}
\usepackage{thmtools}
\usepackage{blindtext}
\usepackage{hyperref}

\newtheorem{theorem}{Theorem}[section]
\theoremstyle{definition}
\newtheorem{definition}{Definition}[section]
\theoremstyle{remark}
\newtheorem*{remark}{Remark}
\newtheorem{corollary}{Corollary}[theorem]
\newtheorem{lemma}[theorem]{Lemma}
\declaretheoremstyle{example}
\declaretheorem[numberwithin=section, style=example, name=Example]{example}
\declaretheoremstyle{proposition}
\declaretheorem[numberwithin=section, style=proposition, name=Proposition]{proposition}

\hypersetup{
    colorlinks=true,
    linkcolor=blue,
    filecolor=magenta,      
    urlcolor=cyan,
    pdftitle={Overleaf Example},
    pdfpagemode=FullScreen,
    }

\urlstyle{same}

\setlength{\droptitle}{-6em}

\title{Class Notes MATH 251}
\author{Alexandre St-Aubin \\ McGill University}
\date{\today}

\begin{document}
\begin{titlepage}
	\centering
	{\scshape McGill University \par}
	\vspace{1cm}
	{\scshape\Large MATH 251\par}
	\vspace{1.5cm}
	{\huge\bfseries Class Notes\par}
	\vspace{2cm}
	{\Large\itshape Alexandre St-Aubin\par}
	\vfill
	Taught by \par
	Mikaël \textsc{Pichot}

	\vfill

% Bottom of the page
	{\large \today\par}
\end{titlepage}

\tableofcontents

\newpage



\section{Fields and Rings}
    \begin{definition}(Ring)
        A set R is a ring if:
        \begin{enumerate}
            \item $a+b= b+a \; \forall a,b \in R$
            \item $(a+b) +c = a+(b+c) \; \forall a,b,c \in R$
            \item $\exists 0 \in R \; s.t. \; a+0= a \; \forall a \in R$
            \item $\forall a \in R,\; \exists (-a) \; s.t. \; a + (-a) = 0$
            \item $(ab)c= a(bc) \; \forall a,b,c \in R$
            \item $a(b+c)= ab+ ac \; \forall a,b,c \in R$
        \end{enumerate}
    \end{definition}
    \begin{definition}(Ring with unity)
        $$1\in R$$
    \end{definition}
    \begin{definition}(Commutative Ring)
        $$ ab = ba \; \forall a, b \in R$$
    \end{definition}
    \begin{definition}(Integral domain)
        If the following hold:
        \begin{enumerate}
            \item $1\in R$
            \item $ ab = ba \; \forall a, b \in R$
            \item $\forall a,b \in R, \; ab=0 \implies a=0 \; or \; b=0$
        \end{enumerate}
        i.e. commutative ring with unity and no zero divisors.
    \end{definition}
    \begin{definition}(Division Ring)
        Ring with unity such that $\forall R\ni a \neq 0$, a is a unit.
    \end{definition}
    \begin{definition}(\textbf{Field})
        A field $\mathbb{F}$ is a commutative ring with unity such that $\forall x\neq 0 \in \mathbb{F}$, x is a unit
    \end{definition}
    \begin{theorem}
        For any prime $p$, $\mathbb{F}_p:=(\mathbb{Z}/p\mathbb{Z},+,\cdot)$ is a field, i.e. $\mathbb{Z}_p$ is a field.
    \end{theorem}

    \begin{definition}(\textbf{Subfield})
        Let $(S,*,\cdot)$ be be an algebraic structure with 2 operations.
        Let T be a subset of S such that $(T,*,\cdot)$ is a field. Then, $(T,*,\cdot)$ is a \textbf{subfield} of S.
    \end{definition}
\section{Vector spaces}
    \begin{definition}(Vector Space)
        A vector space $V$ over a field $\mathbb{F}$ (or $\mathbb{F}$-vector space) is a set on which 
        two operations (addition and scalar multiplication) are defined so that for each pair of elements 
        $x$, $y$  in $V$, there is a unique element $x+y$ in $V$, and for each element $a$ in $\mathbb{F}$ and each element 
        $x$ in $V$ there is a unique element $ax$ in $V$, such that the following conditions hold.
        \begin{itemize}
            \item Additive axioms:
            \begin{enumerate}
                \item For all $x,y$ in $V$, $x+y=y+x$ (commutativity of addition)
                \item For all $x,y,z$ in $V$, $x+(y+z)=(x+y)+z$ (associativity of addition)
                \item There exists an element $0$ in $V$ such that $x+0=0+x=x$, $\forall x\in V$
                \item For each $x\in V$, $\exists y\in V$ such that $x+y = 0$ 
            \end{enumerate}
            \item Multiplicative axioms:
            \begin{enumerate}
                \item For each $x \in V$, $1x = x$
                \item For each $x \in V,\; 0x=0$
                \item For each pair of elements $a,b$ in $\mathbb{F}$, and $x\in V$, $(ab)x=a(bx)$
            \end{enumerate}
            \item Distributive axioms:
            \begin{enumerate}
                \item For each element $a \in \mathbb{F}$ and pair of elements $x,y \in V$, $a(x+y)=ax+ay$
                \item For each pair of elements $a,b \in \mathbb{F}$ and $x \in V$, $(a+b)x = ax+bx$ 
            \end{enumerate}
        \end{itemize}

    \begin{remark}
        The elements of the field $\mathbb{F}$ are called scalars and the elements of the vector 
        space $\mathbb{V}$ are called vectors.
    \end{remark}
    \end{definition}

    \begin{definition}(n-tuple)
    An object of the form $(a_1, a_2, . . . , a_n)$ with entries in $\mathbb{F}$ is 
    called an n-tuple. Two n-tuples are equal if they are equal component-wise.
    \end{definition}

    \begin{definition}($\mathbb{F}^n$)
        The set of all n-tuples with entries from a field $\mathbb{F}$ is denoted $\mathbb{F}^n$. 
        This set is a vector space over $\mathbb{F}$ with operations
        addition and scalar multiplication, and we have that
        \begin{equation*}
            \mathbb{F}^n=\left\{ 
            \begin{pmatrix}
            x_{1}\\x_{2}\\\vdots \\ x_{n}
            \end{pmatrix} 
            :x_i \in \mathbb{F} \right\}
        \end{equation*}
        
    \end{definition}

    \begin{theorem}(Cancellation Law for Vector Addition)
        If $x,y,z$ are vectors in a vector space $V$ such that $x+y= x+z$, then $y=z$. 
    \end{theorem}

    \begin{theorem}
        In any vector space $V$, the following are true:
        \begin{enumerate}
            \item $0x = 0, \forall x\in V$
            \item $(-a)x=-(ax)=a(-x), \forall a \in \mathbb{F}, x\in V$
            \item $a0=0, \forall a \in \mathbb{F}$
        \end{enumerate}
    \end{theorem}
\section{Subspaces}
    \begin{definition}(Subspace)
        A subset W of a vector space V over a field $\mathbb{F}$ is called a subspace of V if W is a vector 
        space over $\mathbb{F}$ with the operations of addition and scalar multiplication defined on V.
        
    \end{definition}
    \begin{remark}
        In any vector space, V and {0} are subspaces. The latter is called the zero subspace of V.
    \end{remark}
    \begin{theorem}
        A subset W of a vector space V is a \textbf{subspace} of V if and only if the following
        properties hold.
        \begin{enumerate}
            \item $x+z \in W$ whenever $x,y \in W$. (W is closed under addition)
            \item $cx \in W$ whenever $c \in \mathbb{F}$ and $x \in W$. (W is closed under scalar multiplication)
            \item W has a zero vector.
        \end{enumerate}
    \end{theorem}
    \begin{theorem}
        Let $U\subseteq V$, with $V$ a vector space over $\mathbb{F}$, TFAE:
        \begin{enumerate}
            \item $U$ is a vector space over $\mathbb{F}$
            \item $\forall n,v \in U, \lambda \in \mathbb{F}$, we have $n+\lambda v \in U$
            \item $\exists S \subseteq U$ such that $U=span(S)$
        \end{enumerate}
    \end{theorem}
    \begin{definition}(Transpose of a matrix)
        The transpose $A'$ of an $m \times n$ matrix A is the $n \times m$ matrix obtained 
        from A by interchanging the rows with the columns.
    \end{definition}
    \begin{definition}(Symmetric matrix)
        A symmetric matrix is a matrix $A'$ such that $A' = A$.
    \end{definition}
    \begin{definition}(Trace of a matrix)
        The trace of an $n \times n$ matrix M, denoted $tr(M)$, is the sum of the diagonal entries of M.
    \end{definition}

    \begin{theorem}
        Any intersection of subspaces of a vector space V is a subspace of V.
    \end{theorem}
\section{Linear Combinations and Systems of linear equations}
    \begin{definition}(Linear Combination)
        Let V be a vector space and S a nonempty subset of V. A vector $v \in V$ is called a 
        \textbf{linear combination} of vectors of S if there exist a finite number of vectors $u_1, u_2,..., u_n
        \in S$ and scalars $a_1, a_2 ,..., a_n \in \mathbb{F}$ such that $v = a_1u_1 + a_2u_2 +...+ a_nu_n$.
        In this case we also say that v is a linear combination of $u_1, u_2,..., u_n$ and call $a_1, a_2,..., a_n$ the \textbf{coefficients}
        of the linear combination.
    \end{definition}

    \begin{definition}(Span)
        Let S be a nonempty subset of a vector space V. The \textbf{span} of S, denoted $span(S)$, 
        is the set consisting of all linear combinations of the vectors in S. 
        For convenience, we define $span(0)={(\emptyset)}$.

    \end{definition}
    \begin{example}
        In $\mathbb{R}^3$, the span of the set $\{(1,0,0), (0,1, 0)\}$ consists of all vectors in 
        $\mathbb{R}^3$ that have the form $a(1,0,0)+b(0,1,0) = (a,b,0)$ for some
        scalars $a$ and $b$. Thus the span of $\{(1,0,0), (0, 1,0)\}$ contains all the points in 
        the xy-plane. In this case, the span of the set is a subspace of $\mathbb{R}^3$. This fact is true in general.
    \end{example}
    \begin{theorem}
        The span of any subset S of a vector space V is a subspace of V. Moreover, any subspace 
        of V that contains S must also contain the span of S.
    \end{theorem}

    \begin{definition}
        A subset S of a vector space V \textbf{generates} (or \textbf{spans}) V if $span(S) = V$. In this case, 
        we also say that the vectors of S generate (or span) V.
    \end{definition}

\section{Linear Dependence and Linear Independence}
    \begin{remark}
        Suppose that V is a vector space over an infinite field and that W is a subspace of V. 
        Unless W is the zero subspace, W is an infinite set. It is desirable to find a "small" 
        finite subset S that generates W because we can then describe each vector in W as a linear 
        combination of the finite number of vectors in S. Indeed, the smaller that S is, 
        the fewer computations that are required to represent vectors in W. 
    \end{remark}
    \begin{definition}(Linear Dependence)
        A subset S of a vector space V is called \textbf{linearly dependent} if there exist a finite 
        number of distinct vectors $u_1, u_2 ,...,u_n \in S$ and scalars $a_1,a_2,...,a_n \in \mathbb{F}$
        not all zero, such that
        \begin{equation*}
            a_1u_1+a_2u_2+...+a_nu_n=0
        \end{equation*}
    \end{definition}

    \begin{definition}(Linear Independence)
        A subset S of a vector space that is not linearly dependent is called \textbf{linearly independent}. 
        As before, we also say that the vectors of S are linearly independent.
    \end{definition}

    \begin{theorem}
        A set is linearly independent $\iff$ the only representations of 0 as linear combinations of 
        its vectors are trivial representations.
    \end{theorem}

    \begin{theorem}
        Let V be a vector space, and let $S_1 \subseteq S_2 \subseteq V$. If $S_1$ is linearly dependent, 
        then $S_2$ is linearly dependent.
    \end{theorem}

    \begin{corollary}
        Let V be a vector space, and let $S_1 \subseteq S_2 \subseteq V$. If $S_2$ is linearly independent, 
        then $S_1$ is linearly independent.
    \end{corollary}

    \begin{theorem}
        Let S be a linearly independent subset of a vector space V, and let v be a vector in V 
        that is not in S. Then $S \cup \{v\}$ is linearly dependent $\iff v \in span(S)$.
    \end{theorem}

\section{Bases and Dimensions}
    \begin{definition}(Basis)
        A \textbf{basis} $\beta$ for a vector space V is a linearly independent subset of V that generates V. 
        If $\beta$ is a basis for V, we also say that the vectors of $\beta$ form a basis for V.
    \end{definition}
    \begin{example}
        In $\mathbb{F}^n$, let $e_1 = (1,0,\hdots,0),e_2 = (0,1,\hdots,0),..., e_n = (0,0,\hdots,1)$. 
        Then, $\{e_1, e_2, \hdots, e_n\}$ is easily seen to be the basis for $\mathbb{F}^n$ and is 
        said to be the standard basis for $\mathbb{F}^n$.
    \end{example}

    \begin{theorem}
        Let V be a vector space over $\mathbb{F}$ and $\beta = {u_1,u_2,... ,u_n}$ be a subset of V. Then $\beta$ is 
        a basis for V $\iff$ each $v \in V$ can be uniquely expressed as a linear combination of vectors 
        of $\beta$, that is, can be expressed in the form
        \begin{equation*}
            v=a_1u_1+a_2u_2+\hdots+a_nu_n
        \end{equation*}
        for unique scalars $a_1, a_2, ..., a_n \in \mathbb{F}$
    \end{theorem}
    \begin{remark}
        Theorem 5.1 shows that if the vectors ${u_1,u_2,... ,u_n}$ form a basis for a vector space V, 
        then every vector in V can be uniquely expressed in the form
        \begin{equation*}
            v=a_1u_1+a_2u_2+\hdots+a_nu_n
        \end{equation*}
        for unique scalars $a_1, a_2, ..., a_n \in \mathbb{F}$. Thus v determines a unique
        $n$-tuple of scalars $a_1, a_2, ..., a_n$ and, conversely, each $n$-tuple of scalars determines
        a unique vector $v \in V$ by using the entries of the $n$-tuple as the
        coefficients of a linear combination of ${u_1,u_2,... ,u_n}$.
    \end{remark}
    \begin{theorem}
        If a vector space V is generated by a finite set S, then some subset of S is a basis for V. 
        Hence V has a finite basis.
    \end{theorem}

    \begin{theorem}(Replacement Theorem)
        Let V be a vector space that is generated by a set G containing exactly n vectors, 
        and let L be a linearly independent subset of V containing exactly m vectors. 
        Then $m<n$ and there exists a subset H of G containing exactly $n-m$ vectors such that 
        $L \cup H$ generates V.
    \end{theorem}
    \begin{corollary}
        Let V be a vector space having a finite basis. Then every basis for V contains the 
        same number of vectors.
    \end{corollary}

    \begin{definition}(Dimension)
    A vector space is called \textbf{finite-dimensional} if it has a basis consisting 
    of a finite number of vectors. The unique number of vectors in each basis for V is called the 
    \textbf{dimension} of V and is denoted by $dim(V)$. A vector space that is not 
    finite-dimensional is called \textbf{infinite-dimensional}.
    \end{definition}

    \begin{example}
        The vector space $\{0\}$ has dimension zero.
    \end{example}
    \begin{example}
        The vector space $\mathbb{F}^n$ has dimension $n$.
    \end{example}
    \begin{example}
        The vector space $M_{m\times n}(\mathbb{F})$ has dimension $mn$.
    \end{example}
    \begin{example}
        The vector space $P_n(\mathbb{F})$ has dimension $n+1$.
    \end{example}

    \begin{corollary}
        Let V be a vector space with dimension $n$, then
        \begin{enumerate}
            \item Any finite generating set for V contains at least $n$ vectors, and a generating 
            set for V that contains exactly $n$ vectors is a basis for V.
            \item Any linearly independent subset of V that contains exactly $n$ vectors is a basis for V.
            \item Every linearly independent subset of V can be extended to a basis for V.
        \end{enumerate}
    \end{corollary}

    \begin{definition}(Line)
        A \textbf{line} is a set of the form $span(v)$, $v \in V$ such that
        $v\neq 0$, i.e. $dim_K (span(v))=1$. Note that $0\in span(v)$, simply by choosing $\lambda =0$. In $\mathbb{F}_2^n$, there are $2^n-1$ lines. 
    \end{definition}
    \begin{definition}(Plane)
        A \textbf{plane} is a set of the form $span(u,v)$, with $u,v \in V$ such that $u,v\neq 0$, $u,v$ L.I., 
        i.e. $dim_K (span(u,v))=2$. Note that $0\in span(u,v)$.
    \end{definition}
    \begin{theorem}
        Let V be a vector space with dimension n.
        \begin{enumerate}
            \item Any finite generating set for V contains at least n vectors, and a gener-
            ating set for V that contains exactly n vectors is a basis for V.
            \item Any linearly independent subset ofV that contains exactly n vectors is a basis for V.
            \item Every linearly independent subset of V can be extended to a basis for V.
        \end{enumerate}
    \end{theorem}
    \begin{theorem}(\textbf{Major Theorem on Basis})
        Let V a finite dimension vector space (V f.d. v.s.) over $\mathbb{F}$, where $v_1, \hdots, v_n$
        are basis vectors in V. Then, TFAE:
        \begin{enumerate}
            \item $v_1, \hdots, v_n$ is a minimal spanning set, i.e. $\nexists$ a spanning set of $<n$ elements.
            \item $v_1, \hdots, v_n$ is a maximal L.I. (linearly independent) set. (see definition 6.3)
            \item $v_1, \hdots, v_n=$ spanning + L.I.
            \item Any $v\in V$ can be written as a linear combination $v=\sum \lambda_iv_i$ where the 
            $\lambda_i$'s (called the \textbf{coordinates} of v in basis $v_1, \hdots, v_n$) are unique and $1<i<n$. 
        \end{enumerate}
    \end{theorem}
    \begin{corollary}
        Any two bases are the same size, i.e. if both ${u_1,u_2,... ,u_n}$ and ${u_1,u_2,... ,u_m}$ satisfy (3), then $m = n$.
    \end{corollary}
    \begin{lemma}
        Suppose $V=span(v_1, \hdots, v_n)$, Then, any ${u_1,u_2,... ,u_{n+1}}$ are L.D. 
    \end{lemma}

    \begin{theorem}(Basis Extension Theorem)
        Every linearly independent list of vectors in a
        finite-dimensional vector space V can be extended to a basis of V. That is, suppose you have 
        a v.s. V, $(u_1,...,u_n)$ L.I. vectors and $(v_1,...,v_m)$ span(V). Then, $\exists (u_{n+1},...,u_k) \in \{v_1,...,v_m\}$
        such that $(u_1,...,u_k)$ is a basis of V.
        \begin{proof}
            Suppose V is finite-dimensional and $(u_1,...,u_n)$ is linearly independent. Since V is
            finite-dimensional, there exists a list $\{v_1,...,v_m\}$ of vectors that spans V . We wish to adjoin
            some of the $v_m$ to $(u_1,...,u_n)$ to create a basis of V .
            \newline
            \textbf{Step 1.} If $v_1 \in span(u_1,...,u_n)$, let $S=(u_1,...,u_n)$, otherwise, $S=(u_1,...,u_n,v_1)$.
            \newline 
            \textbf{Step k.} If $v_k \in span(S)$, leave S unchanged, otherwise, adjoin $v_k$ to S.
            \newline
            After each step the list S is still linearly independent since we only adjoined $v_k$ if $v_k$ was
            not in the span of the previous vectors. After n steps $v_k \in span(S) \forall k = 1, 2, . . ., m$.
            Since $(v_1, . . ., v_m)$ was a spanning list, S spans V , so that S is indeed a basis of V .
        \end{proof}
        
    \end{theorem}

    \begin{corollary}
        $U\subseteq V \implies dim(U)\le dim(V)$
        \begin{proof}
            Simply extend a basis of U to a basis of V (basis ext. thm).
        \end{proof}
    \end{corollary}

    \begin{proposition}(Finding a basis)
        Let $v_1,..., v_m \in R^n$. We want to find a basis of the span of $v_1,..., v_m$. 
        \begin{enumerate}
            \item Write the vectors as rows on a matrix. 
            \item Use Gaussian algorithm to reduce to REF. 
            \item Then, $span(v_1,..., v_m) = span$(rows of the REF), i.e. discard the rows of 
            zeros as they are L.D.
        \end{enumerate}
    \end{proposition}
\section{Maximal Linearly Independent Subsets}
    \begin{definition}(Maximal)
        Let $\tau$ be a family of sets. A member M of $\tau$ is called \textbf{maximal} (with respect to set inclusion) if M is contained in no 
        member of $\tau$ other than M itself.
    \end{definition}

    \begin{definition}(Chain)
        A collection of sets C is called a \textbf{chain} (or nest or tower) if for each pair of sets A and B in C, either $A\subseteq B$ or $B \subseteq A$.
    \end{definition}

    \begin{lemma}(\textbf{The Maximal Principle})
        Let $T$ be a family of sets. If, for each chain $C\subseteq T$, there exists a member of $T$ that 
        contains each member of C, then $T$ contains a maximal member.
    \end{lemma}

    \begin{definition}(Maximal Linearly Independent Subset)
        Let S be a subset of a vector space V. A \textbf{maximal linearly independent subset} of S is a subset 
        B of S satisfying both of the following conditions.
        \begin{enumerate}
            \item B is linearly independent.
            \item The only linearly independent subset of S that contains B is B itself.
        \end{enumerate}
    \end{definition}

    \begin{theorem}
        Let V be a vector space and S a subset that generates V. If $\beta$ is a maximal 
        linearly independent subset of S, then $\beta$ is a basis for V.
    \end{theorem}

    \begin{theorem}
        Let S be a linearly independent subset of a vector space V. There exists a 
        maximal linearly independent subset of V that contains S.
    \end{theorem}
    \begin{corollary}
        Every vector space has a basis.
    \end{corollary}

\section{Sums}
    Let $W$ a $v.s.$, with $U,V\subseteq W$ subspaces.
    \begin{definition}
       $U+V:=\{u+v:u\in U, v\in V\} $
    \end{definition}
    \begin{definition}(Direct Sum)
        We say that $W=U\oplus V$ if
        \begin{enumerate}
            \item $W=U+V$
            \item $U\cap V =\{0\}$
        \end{enumerate}
    \end{definition}
    \begin{proposition}
        $W=U\oplus V \iff$ any $w \in W$ can be written as $w=u+v$ in a unique way ($u\in U, v\in V$).
    \end{proposition}
    \begin{remark}
        Let $v_1,...,v_n$ a basis of V, then $V=\{Kv_1\oplus...\oplus Kv_n\}$, where $Kv_1= span(v_1)=$ a line generated by $v_1$.
    \end{remark}
    \begin{theorem}(Direct Sums and Dimension)
        Suppose U and V are subspaces of W, with $U\oplus V=W$, then $dim(W)=dim(U)+dim(V)$.
    \end{theorem}
    \begin{theorem}(\textbf{Gluing Basis})
        Let $E_i=(e_{1,i},...,e_{k,i})$ be a basis of $V_i$. Then $E_1\sqcup...\sqcup E_n$ is a 
        basis of $V_1\oplus...\oplus V_n$.
        
    \end{theorem}
    \begin{definition}(Complement)
        Let $U \subseteq W$, we say $V\subseteq W$ is a complement of $U$ if $U\oplus V = W$.
    \end{definition}
    \begin{proposition}
        The complement always exists.
        \begin{proof}
            Choose a basis of U $(u_1,...,u_n)$, extend it to a basis of W $(u_1,...,u_n, u_{n+1}, ...,u_m)$. 
            Then, let $V=span(u_{n+1}, ...,u_m)$, so we have that $V=$complement of U.
        \end{proof}
    \end{proposition}
    \begin{proposition}
        If $U_1$ and $U_2$ are subspaces of a finite dimensional vector space then: 
        $$\dim(U_1+U_2)=\dim U_1 + \dim U_2 - \dim(U_1 \cap U_2)$$       
        \begin{proof}
            Let $u_1,...,u_m$ be a basis of $U_1\cap U_2$; thus $\dim (U_1\cap U_2)=m$. 
            Because $u_1,...,u_m$ is a basis in $U_1\cap U_2$, it is linearly independent in $U_1$. 
            Hence this list can be extended to a basis $u_1,...,u_m,v_1,...,v_j$ of $U_1$. 
            Thus, $\dim U_1 = m+j$. Also extend $u_1,..,u_m$ to a basis $u_1,...,u_m, w_1,...,w_k$ of 
            $U_2$. $\dim U_2 = m +k$. We will show that $u_1,...,u_m,v_1,...,v_j,w_1,...,w_k$ is a basis 
            of $U_1+U_2$. This will complete the proof because then we will have 
            $$\dim(U_1+U_2)=m+j+k=\dim U_1 + \dim U_2 - \dim (U_1\cap U_2)$$

            We just need to show that the list $u_1,...,u_m,v_1,...,v_j,w_1,...,w_k$ is linearly independent. 
            To prove this, suppose: 

            $$a_1u_1+...+au_m+b_1v_1+...+b_jv_j+c_1w_1+...+c_kw_k=0$$

            where all $a,b,c$'s are scalars. We need to show that all the $a,b$ and $c$'s are $0$.

            The equation can be rewritten as 

            $$c_1w_1+...+c_kw_k=-a_1u_1 - ... -a_mu_m -b_1v_1 - ... - b_j v_j$$

            Which shows that $c_1w_1+...+c_kw_k\in U_1$. But actually all $w$'s are in $U_2$. So the LHS 
            must be an element of $U_1\cap U_2$.

            $c_1w_1+...+c_kw_k=d_1u_1+...+d_mu_m$ for some choice of scalars $d_1,d_2,...,d_m$. 
            But $u_1,...,u_m,w_1,...,w_k$ is linearly independent. So our last equation implies 
            that all the $c$'s equal $0$.

            Thus our original equation involving $a,b,c$ becomes

            $$a_1u_1+...+a_mu_m+b_1v_1+...+b_jv_j=0$$

            But we already knew that the list $u_1,...,u_m,v_1,...,v_j$ is linearly independent. 
            This equation implies that all the $a$'s and $b$'s are $0$. We now know that all 
            $a,b$ and $c$'s are $0$, hence proving our original claim.
        \end{proof}
        \textbf{Second Proof}
        \begin{proof}
            let $$U=(U\cap V)\oplus U'$$ $$V=(U\cap V)\oplus V'$$ where $U'$ is the complement 
            of $(U\cap V)$ inside $U$, and $V'$ the complement of $(U\cap V)$ inside $V$, which both
            exist by Prop 8.2. Hence, we have that 
            $$U+V = (U\cap V)\oplus U' \oplus V'$$
            $$dim(U+V) = dim(U\cap V)\oplus dim(U') \oplus dim(V')$$
            $$dim(U+V) = dim(U\cap V) + dim(U-(U\cap V)) + dim(V-(U\cap V))$$
            $$dim(U+V) =  dim(U) + dim(V)-dim(U\cap V)$$
            Must also show that $$(U\cap V) \cap (U'\cap V') = 0$$
            Let $u\in (U\cap V)\cap(U'+V')$. Then, $u\in (U'+V')\implies u=u'+v'\implies v'\in V' \implies v'=u-u' \in (U\cap V)\implies v'=0$
            \newline 
            And, show $$U'\cap (U\cap V+V') = 0$$
            Finally, $$V'\cap (U\cap V+U') = 0$$
        \end{proof}
    \end{proposition}
\section{Linear Maps}
    \begin{definition}(Linear Map)
        Let U, V be vector spaces. Then, $f: U \rightarrow V$ is a linear map if:
        \begin{enumerate}
            \item $f(U+V)=f(U) + f(V)$
            \item $f(\lambda U)=\lambda f(U)$
        \end{enumerate}
        $\iff f(\sum \lambda_i u_i)=\sum \lambda_i f(u_i)$
    \end{definition}
    \begin{example}
        Let $f:K^n \rightarrow V$, $v_1, ..., v_n \in V$ then, 
        $$\begin{pmatrix}
            \lambda_1\\ \vdots \\ \lambda_n 
        \end{pmatrix} \rightarrow \sum_i \lambda_i v_i$$
        is a linear map.
    \end{example}
    \begin{theorem}
        Let V and W be vector spaces over F, and suppose that ${v_i,v_2,...,v_n}$ is a basis for V.
        For $w_1,w_2,..., w_n$ in W ,there exists exactly one linear transformation $T: V\rightarrow W$ such that $T(v_i) = w_i$ for  $i = 1,2,...,n$ 
    \end{theorem}
    \begin{definition}(Isomorphism)
        An isomorphism is a map which is bijective and linear.
    \end{definition}
    \begin{proposition}
        The following are equivalent:
        \begin{enumerate}
            \item $f$ is an isomorphism. 
            \item $u_1,...,u_n$ is a basis of V.
        \end{enumerate}
        \begin{proof}
            Suppose (1), $v_1, ..., v_n$ span, so f is surjective. 
            $$\forall v \in V, \exists \begin{pmatrix}
                \lambda_1\\ \vdots \\ \lambda_n 
            \end{pmatrix} \in \mathbb{R}^n \: s.t. \: f\begin{pmatrix}
                \lambda_1\\ \vdots \\ \lambda_n 
            \end{pmatrix} = V \implies V=\sum \lambda_i v_i$$
            L.I.: Assume $\sum \lambda_i v_i = 0$, then 
            $$ f\begin{pmatrix}
                \lambda_1\\ \vdots \\ \lambda_n 
            \end{pmatrix}=0=f\begin{pmatrix}
                0\\ \vdots \\ 0 
            \end{pmatrix} \implies \lambda_i = 0 \; \forall i $$
            So, $v_1, ..., v_n$ is a basis. 
            \newline
            Now, suppose (2), we show (1). Clearly, f is linear. Also, f is injective: suppose
            $$f\begin{pmatrix}
                \lambda_1\\ \vdots \\ \lambda_n 
            \end{pmatrix} = f\begin{pmatrix}
                \mu_1\\ \vdots \\ \mu_n 
            \end{pmatrix}$$
            Then, $\sum \lambda_i v_i = \sum \mu_i v_i $, so $\lambda_i = \mu_i$. 
            \newline
            f is surjective: take $v\in V, \; \exists \begin{pmatrix}
                \lambda_1\\ \vdots \\ \lambda_n 
            \end{pmatrix} \in K^n \; s.t. f\begin{pmatrix}
                \lambda_1\\ \vdots \\ \lambda_n 
            \end{pmatrix} = v$ since $v_1, ..., v_n$ spans V. 
        \end{proof}
    \end{proposition}
    \begin{corollary}
        Every finite dimension vector space is isomorphic to $K^n$ for some n.
    \end{corollary}

    \begin{example}
        Let $P_n(K)$ be the set of polynomials of degree $\leq n$. Then, $$P_n(K) \cong K^{n+1}$$

        $$a_nx^n +...+a_1x+ a_0 \iff \begin{pmatrix}
            a_n \\ \vdots \\ a_0 
        \end{pmatrix}$$ 
        $$\phi : K^{n+1}\rightarrow P_n(K)$$
        $$\begin{pmatrix}
            a_n \\ \vdots \\ a_0 
        \end{pmatrix} \rightarrow \sum a_k X^K \iff (1, x, x^2,..., x^n) \: is \; basis$$
    \end{example}
    \begin{definition}(Kernel)
        $ker(f) = \{u\in U : f(u)=0\}$ is a subspace of U.
        
    \end{definition}

    \begin{theorem}
        Let $f: U\rightarrow V$ be linear. Then, $$f \; is \; surjective \iff Im(f) =V$$

        $$f \; is \; injective \iff ker(f)= 0$$
        \begin{proof}
            (injective$\implies ker(f) =0$): 
            $$f(u)=0=f(0)\implies u=0$$
            \newline($ker(f) =0 \implies$ injective): 
            \begin{align*}
            ker(f) = 0, \; f(u)= f(v) &\implies f(u)-f(v)=0 \\
                                      &\implies f(u-v)= 0 \; \; \; [since \; f \; is \; linear] \\
                                      &\implies (u-v) \in ker(f) \implies u=v
            \end{align*}
        \end{proof}

    \end{theorem}
    \begin{lemma}
        Let $f:U\rightarrow V$ be linear, let $U = ker(f)\oplus U'$, where U' is the complement of $ker(f)$
        Then, $f|_{U'}$ is injective.
        \begin{proof}
            Suppose $f(u)=f(v)$, with $u,v \in U'$. Then, $u-v\in (U' \cap ker(f))\implies u=v$
        \end{proof}
    \end{lemma}
    \begin{corollary}
        $f: U' \rightarrow f(U)$ is an isomorphism.
        \begin{proof}
            $$f:U' \rightarrow f(U')=f(U)$$
        \end{proof}
    \end{corollary}
    \begin{theorem}
        Let $f:U\rightarrow V$ be a linear map, U, V finite dimension. Then, 
        $$ dim(U) = dim(ker(f))+ dim(Im(f))$$

        \begin{proof}
            Let $U= ker(f) \oplus U'$. Since $f: U'\rightarrow  f(U)$
            is an isomorphism, $$dim(U')  = dim (f(U))= dim (Im(f))$$
            $$ \implies dim(U)= dim (ker(f)) + dim (U') = dim(ker(f)) + dim (Im(f))$$
        \end{proof}
    \end{theorem}
    \begin{corollary}
        Suppose $f: U\rightarrow U$ is a linear map (U f.d.),
        TFAE: 
        \begin{enumerate}
            \item f is an isomorphism. 
            \item f is injective. 
            \item f is surjective. 
        \end{enumerate}
        \begin{remark}
            A linear map from a vector space to itself is called an \textbf{Operator}.
        \end{remark}
        \begin{remark}
            Every linear transformation can be represented by a matrix
            $$ \begin{pmatrix}
                x_1 \\ \vdots \\x_n
            \end{pmatrix} \rightarrow A \begin{pmatrix}
                x_1 \\ \vdots \\x_n
            \end{pmatrix} $$
        \end{remark}
    \end{corollary}
\subsection{Null space (nullity) and Range (rank)}
    \begin{definition}
        Let V and W be vector spaces, and let $T: V \rightarrow W$ be linear. We define the \textbf{null space} 
        (or \textbf{kernel}) $N(T)$ of T to be the set of all vectors x in V such that $T(x) = 0$; that is, $N(T) = { x \in V : T(x) = 0}$.
        We define the \textbf{range} (or \textbf{image}) $R(T)$ of T to be the subset of W consisting of all images (under T) of vectors in V;
         that is, $R(T) = {T(x): x \in V}$.
    \end{definition}
    \begin{theorem}
        Let V and W be vector spaces, and let $T: V\rightarrow W$ be linear. If $\beta = \{v_i,v_2,..., v_n \}$ 
        is a basis for V, then
        $$ Range(T)= span(T(\beta))$$
    \end{theorem}
    
    \begin{definition}
        Let V and W be vector spaces, and let $T: V \rightarrow W$ be linear. If $N(T)$ and $R(T)$ are finite-dimensional, 
        then we define the nullity of T, denoted $nullity(T)$, and the rank of T, denoted $rank(T)$, to be the dimensions 
        of $N(T)$ and $R(T)$, respectively.
    \end{definition}
    \begin{theorem}
        Let V and W be vector spaces and $T: V \rightarrow W$ be linear. Then $N(T)$ and $R(T)$ are subspaces of $V$ and $W$, respectively.
    \end{theorem}
    \begin{theorem}(Dimension Theorem)
        Let V and W be vector spaces, and let $T: V \rightarrow W$ be linear. If V is finite-dimensional, then
        $$nullity(T) + rank(T) = dim(V)$$

    \end{theorem}
\subsection{Matrix Representation}

    \begin{definition}(\textbf{Matrix Representation})
        We call the $m \times n$ matrix A defined by $A_{ij} = a_{ij}$ the matrix representation of T in the ordered 
        bases $\beta$ and $\gamma$ and write $A=[T]^\gamma_\beta$.If $V=W$ and $\beta=\gamma$, then we write 
        $A=[T]_\beta$.
        
        Notice that the $j$th column of A is simply $[T(v_j)]_\gamma$. Also observe that if
        $U: V \rightarrow W$ is a linear transformation such that $[U]^\gamma_\beta = [T]^\gamma_\beta$, then $U = T$.
    \end{definition}
    \begin{definition}(Kronecker Delta)
        We define the Kronecker delta $\delta_{ij}$ by $\delta_{ij} = 1$ if $i=j$ and
        $\delta_{ij} =0$ if $i\neq j$. The $n \times n$ identity matrix $I_n$ is defined by $(I_n)_{ij} = \delta_{ij}$.
    \end{definition}
    \begin{theorem}
        Let V and W be finite-dimensional vector spaces with ordered bases $\beta$ and $\gamma$, respectively, and let $T, U: V \rightarrow W$ be linear transformations. Then
        \begin{enumerate}
            \item $[T+U]^\gamma_\beta=[T]^\gamma_\beta +[U]^\gamma_\beta$.
            \item $[aT]^\gamma_\beta = a[T]^\gamma_\beta$.
        \end{enumerate}

    \end{theorem}
    \begin{definition}
        Let A be an $n\times n$ matrix with entries from a field F. We denote by 
        $L_A$ the mapping $L_A: F^n\rightarrow F^m$ defined by $L_A(x) = Ax$ (i.e. the matrix 
        product of x and A) for each column vector $x\in F$. We call $L_A$ a \textbf{Left-multiplication Transformation}.
    \end{definition}
    \subsection{Scaling transformation}
    \begin{definition}(Scaling transformation)
        $ T: U \rightarrow U$ is a scaling transformation if $\exists (v_1, ..., v_n)$ basis of $U$
        and $\lambda_1, ... \lambda_n \in K$ s.t. $T(v_i) = \lambda_i v_i$. 
    \end{definition}
    \subsection{Projection}
    \begin{definition}(Projection)
        Let V a v.s., with $V = V_1\oplus V_2 $ be a direct sum decomposition, 
        $ \forall  v\in V,\exists! v_1 \in V_1, v_2 \in V_2 \; s.t. \; v= v_1 + v_2$. The projection of v onto $V_1$ is 
        defined as taking only the $v_1$ element of v. It is a linear transformation. 
        \begin{proof}
            Let $P$ be a projection, let $u=u_1 + u_2$, $v= v_1+ v_2$, then
            $$u+v =  (u_1 + u_2) +(v_1+ v_2) $$
            $$P(u+v)= P[(u_1 + u_2) +(v_1+ v_2)]$$
            $$ P(u+v) = P[(u_1 +v_1) +(u_2 + v_2)]$$
            $$ P(u+v) = u_1 + v_1$$
            $$ P(u+v) = P(u)+P(v)$$

        And, $$P(\lambda u) = P(\lambda( u_1 +u_2))$$
        $$P(\lambda u) = P(\lambda u_1 + \lambda u_2)$$
        $$P(\lambda u) = \lambda u_1 = \lambda P(u_1 + u_2) =\lambda P(u)$$
        \end{proof}
    \end{definition}
        \begin{proposition}
            P is a projection $ \iff \; P = P^2$
            \begin{proof}
                For the forward implication, suppose $P: V \rightarrow V$ is a projection on the subspace W. Then, let
                $v \in V$, we have $P(v) = w_1 \in W$, so $P^2(v) = P(P(v)) = P(w_1) = w_1 = P(v)$ since $P(w)=w, \; \forall w \in W$
                \newline
                \newline
                Now, for the other implication, suppose we have a linear operator $P: V \to V$ such that $P^2=P$. Since $P$ is a linear operator, 
                we have a decomposition of $V$ into a direct sum
                $$V= \operatorname{ker}P \oplus \operatorname{im}P$$
                because 
                \begin{enumerate}
                    \item Let $v\in ker(P)\cap Im(P)$, then $v=P(u)$ and $P(v)=0$ (because it's in the kernel), but then, 
                    since $P=P^2$, we get $ 0=P(v)=P^2(u)= P(u)=0$, so $ker(P)\cap Im(P) = \{0\}$.

                    \item We show $ker(P) + Im(P)$ spans V. Let $v\in V$, $P(v) \in Im(P)$. Let's show 
                    $V = (V-P(V))+P(v)$, then $(V-P(V))$ must be in $ker(P)$. 
                    $$P(V-P(V))=P(V) - P^2(U) = P(V)-P(V)=0$$
                    So, $(V-P(V)) \in ker(P)$ and we have shown that $V= \operatorname{ker}P \oplus \operatorname{im}P$.
                \end{enumerate}
                We show that $P$ is a projection onto the subspace $\operatorname{im}P$. By definition, 
                $P(V)=\operatorname{im}P$. Secondly, for any $w$ in the image of $P$ we have $P(w)=P(P(v))$ for some 
                $v \in V$. But by assumption $P^2=P$, so $P(w)=P(v)=w$. Hence $P$ restricts to the identity on 
                $\operatorname{im}P$ and so $P$ is a projection operator. 
            \end{proof}
        \end{proposition}
        \begin{theorem}(Complementarity of image and kernel)
            
        Let $W$ be a finite-dimensional vector space and $P$ be a projection on $W$. Suppose the subspaces $U$ and $V$ are the image and kernel of $P$ respectively. Then $P$ has the following properties:

        \begin{enumerate}
            \item $P$ is the identity operator $I$ on $U$: $\forall \mathbf x \in U: P \mathbf x = \mathbf x$.
            
            \item we have a direct sum $W = U \oplus V$. Every vector $\mathbf x \in W$ may be decomposed uniquely as $\mathbf x = \mathbf u + \mathbf v$ with $\mathbf u = P \mathbf x$ and $\mathbf v = \mathbf x - P \mathbf x = \left(I-P\right) \mathbf x$, and where $\mathbf u \in U, \mathbf v \in V.$
            
        \end{enumerate}
        The image and kernel of a projection are "complementary", as are $P$ and $Q = I - P$. The operator $Q$ is also a projection as the image and kernel of $P$ become the kernel and image of $Q$ and vice versa. We say $P$ is a projection along $V$ onto $U$ (kernel/image) and $Q$ is a projection along $U$ onto $V$.
        \end{theorem}
        \subsection{Rotation}
        \begin{definition}
            A rotation is a linear transformation $R: \mathbb{R}^2 \rightarrow \mathbb{R}^2$ such that $R(v)$ rotates the 
            vector v by $\theta$ degrees. The transformation matrix of a rotation is:
            $$ R_\theta = \begin{pmatrix}
                cos\theta & -sin\theta \\ sin\theta & cos\theta
            \end{pmatrix}$$
            where $T(e_1) =\begin{pmatrix}
                cos\theta \\ sin\theta 
            \end{pmatrix}$ and $T(e_2) =\begin{pmatrix}
                -sin\theta \\ cos\theta 
            \end{pmatrix}$, with $(e_1,e_2)$ forming a basis of $\mathbb{R}^2$

        \end{definition}
        \begin{remark}
            $\{\mathbb{R}^2\}$ forms a group. 
            $$R^{-1}_\theta = R_{-\theta}$$
            $$ R_{\theta'}R_{\theta}= R_{\theta'+\theta}$$
        \end{remark}
        \begin{definition}
            You can define the group $SO(2)$ as $2\times 2$ matrices:
            $$SO(2) = \left\{\begin{pmatrix}\cos(\theta) & -\sin(\theta) \\ \sin(\theta) & \cos(\theta)\end{pmatrix} : \theta\in \mathbb{R}\right\}.$$
            So the $2$ comes from the fat that you have $2\times 2$ matrices.
            In general
            \begin{align}
            O(n) &= \{A \in GL(n): A^TA = A^TA = I\}. \\
            SO(n) &= \{A \in O(n): det(A) = 1 \}.
            \end{align}
            
            will be a group of $n\times n$ matrices.
            Note that all the matrices above are orthogonal and have determinant $1$.
        \end{definition}
        \begin{proposition}
            The map
            $$\pi :(\mathbb{R},+)\rightarrow (SO(2),\times)\subseteq M_2(\mathbb{R})$$
            defined by $$\pi: \theta \rightarrow R_\theta$$
            is a surjective homomorphism with $ker(\pi) = 2 \pi \mathbb{Z}$. It follows, from the first isomorphism theorem,
            that $$\mathbb{R}/2\pi \mathbb{Z} \cong SO(2)$$

        \end{proposition}

\subsection{Differential operators}
        Let $V=K[x]=$ the set of all polynomials with coefficients in K. Let $f\in V$ be of the form $f= \sum a_n x^n$.
        \begin{definition}
            $$D(f):= \sum na_n x^{n-1} := f'$$
            D defines a linear map $P_n \rightarrow P_n$. $X^n \notin Im(D)$ because we lose a degree when taking the derivative. 
            $\implies$ D is not surjective
        \end{definition}
        \begin{proposition}
            $D: P_n \rightarrow P_n$ is \textbf{nilpoltent}, that is, $\exists N\in \mathbb{N} \; s.t. \; D^n = 0$
        \end{proposition}


\section{Linear forms}
        Let V a vector space over K. 
        \begin{definition}
            A \textbf{linear form} on V is a linear map $f: V\rightarrow K$ (K being the vector space on which V is defined)
        \end{definition}
        \begin{definition}(\textbf{Coordinate vector})
            Let $\beta = \{u_1,u_2,... ,u_n\}$ be an ordered basis for a finite-dimensional vector space V. For $x \in V$, 
            let $(a_1, a_2 ,..., a_n)$ be the unique scalars such that
            $$x = \sum^n_{i=1}a_i u_i $$
            We define the coordinate vector of x relative to $\beta$, denoted $[x]_\beta$, by
            $$[x]_\beta = \begin{pmatrix}
                a_1 \\a_2 \\\vdots \\ a_n 
            \end{pmatrix}$$


        \end{definition}
        \begin{example}
            Let $v_1,..., v_n$ a basis of V $\implies$ any $v \in V$  can be uniquely written as $\sum \lambda_i v_i,\; i \in \{1,...,n\}$
            \begin{proposition}
                $f: V\rightarrow K$ defined by $f:v_i \rightarrow \lambda_i$ is a linear form. 
                \begin{proof}
                    let $v= \sum \lambda_i v_i , \; v'=\sum \lambda'_i v_i$
                        $$\implies v+v' = \sum (\lambda_i+ \lambda'_i)v_i$$
                        $$f(v+v')= coordinate \; i \; of \; v+v' = \lambda_i + \lambda'_i =f(v) + f(v')$$
                        And,
                        $$f(\lambda v)=\lambda f(v)$$

                \end{proof}
            \end{proposition}
        \end{example}
        \begin{proposition}
            Let V a vector space over K. Then, the set of all linear forms on V is a vector space over K. It is called the 
            \textbf{Dual Space} and is denoted $V^*$
            $$V^* = L(V, \mathbb(F)) = \text{space of linear transformations }f: V\rightarrow \mathbb{F}$$
            If $Dim(v) \leq \infty$ then $V\cong V^*$. 
            \begin{proof}
                $$Dim(V^*) = Dim(L(V, \mathbb{F}))$$
            and since 
            $$Dim (L(V,W)) = Dim(V) Dim(W),$$
            we have 
            $$Dim(V^*) = Dim(V) Dim(\mathbb{F}) = Dim(V)\times 1$$
            So, $V$ \& $V^*$ have the same dimension hence are isomorphic.
            \end{proof} 
        \end{proposition}

        \begin{proposition}
            Let $v_1, ... , v_n$ be a basis of V, let 
            \begin{align*}
                f_i:& V \rightarrow K\\
                & v \rightarrow \lambda_i
            \end{align*}
            with $v = \sum \lambda_i v_i$, $1 \leq i \leq n$. Then, $f_1, ..., f_n$ is a basis of $V^*$. 

        \end{proposition}

        \begin{corollary}
            If V is finite dimension, then $dim(V) = dim(V^*)$ $(V\cong V^*)$. This isomorphism depends on the basis for V. 
            $$V \cong V^* \cong V^{**}$$
        \end{corollary}
        \begin{proposition}
            There exists a natural homomorphism 
            $$V\rightarrow V^{**}$$
            that is, the isomorphism is constructed from V itself, without a basis. 
        \end{proposition}
        \begin{definition}(\textbf{Dual Basis})
            We call the ordered basis $\beta^* = {f_1,f_2,\hdots  ,f_n}$ of $V^*$ that satisfies $f_i(x_j) = \delta_{ij} (1 < i,j < n)$ the dual basis of $\beta$, where $\delta_{ij}$ is the Kronecker delta.

        \end{definition}
        \begin{theorem}
            Let $T:V \rightarrow W$ a linear transformation, $\beta =\{v_1, ...,v_n \}$ a basis of V and $\gamma = \{ w_1, ..., w_m\}$ a basis of W.
            Then we have a matrix 
            $$A = [T]^\gamma_\beta$$
            Let $\beta^* =\{f_1, ...,f_n \}$  be the dual basis of V (i.e. basis of $V^*$) and $\gamma^* = \{ g_1, ..., g_m\}$ the basis of $W^*$.
            Then, 
            $$[T^t]^{\beta^*}_{\gamma^*} = A^t$$ 
            \begin{proof}
            Consider $$[T^t]^{\beta^*}_{\gamma^*} = 
\bordermatrix{~ & T^t(g_1)&\hdots T^t(g_j)\hdots &T^t(g_m)\cr
              f_1  & & &\cr
               \vdots & & & \cr
              f_n  & &  & \cr}$$
            And 
            $$[T^t]^{\gamma}_{\beta} = 
\bordermatrix{~ & &T(v_i )&\cr
              w_1  & & &\cr
               \vdots & & & \cr
               w_k & &A_{ki} & \cr
               \vdots & & & \cr
              w_m  & &  & \cr}$$ 
            Recall that if $f\in V^*$, $$f = \sum^n_{i=1} f(v_i) f_i$$
            And $$T^t (g_j) \in V^* = \sum^n_{i=1} T^t(g_j)(v_i)f_i = \sum^n_{i=1} g_j(T(v_i))f_i$$
            Now, 
            \begin{align*}
                T(v_i) &= \sum^m_{k=1} A_{ki}w_k \implies g_j(T(v_i)) = g_j(\sum^m_{k=1} A_{ki} W_k) = \sum^m_{k=1} A_{ki}g_j(w_k)\\
                       &A_{1i} g_j(w_1) + A_{2i}g_j(w_2) + \hdots + A_{ji}g_j(w_j) + \hdots + A_{mi} g_j (w_m)
            \end{align*}
            And since the $g_j$'s are dual basis vectors, they are equal to 1 precicely at $w_j$ and 0 elsewhere, we get
            $$T(v_i) = A_{ji}$$
            To summarize, 
            $$T^t(g_j) = \sum^n_{i=1} A_{ji} f_i = \sum^n_{i=1}(A^t)_{ij} f_i$$
            so $$[T^t]^{\beta^*}_{\gamma^*} = A^t$$
            \end{proof}

        \end{theorem}
\section{Change of Basis}
\begin{theorem}
    Let $\beta$ and $\beta'$ be two ordered bases for a finite-dimensional vector space V, and let $Q = [I_v]^{\beta}_{\beta'}$. Then
\begin{enumerate}
    \item Q is invertible. 
    \item For any $v \in V$, $[v]_\beta= Q[v]_{\beta'}.$
\end{enumerate}
The matrix $Q = [I_v]^{\beta}_{\beta'}$ defined in the above Theorem is called a change of coordinate matrix.
 Because of part (2) of the theorem, we say that Q changes $\beta'$-coordinates into $\beta$-coordinates. 
 Observe that if $\beta = \{x_1,x_2,...,x_n\}$ and $\beta' = \{x'_1,x'_2,...,x'_n\}$, then
 $$x'_j= \sum^n_{i=1} Q_{ij}x_i \quad \text{ for } j=1,2,...,n$$
 That is, the $j$th column of the matrix Q is $[x'_j]_\beta$, i.e. the coordinate vector of $x'$ in basis $\beta$. Moreover, notice that if $Q$ changes $\beta'$-coordinates 
 into $\beta$-coordinates, then $Q^{-1}$ changes $\beta$-coordinates into $\beta'$-coordinates.
\end{theorem}
\begin{example}
    Let $e_1 = \begin{pmatrix}
        1 \\ 0
    \end{pmatrix} , \; e_2 = \begin{pmatrix}
        0 \\ 1
    \end{pmatrix}$ 
    be a canonical basis in $\mathbb{R}^2$. Let $v = \begin{pmatrix}1 \\ 1 \end{pmatrix} = $ coord. of v in e-basis. 
    Take $f_1 = \begin{pmatrix}
        1 \\ 0
    \end{pmatrix},  f_2 = \begin{pmatrix}
        1 \\ 1
    \end{pmatrix}$ be a basis of $\mathbb{R}^2$. Then, the coordinates of v in f-basis are  $\begin{pmatrix}
        0 \\1
    \end{pmatrix}$ because $v = 0\times (f_1) + 1\times (f_2)$
\end{example}
\begin{example}
    Let $f_1 = \begin{pmatrix}
        1\\2\\3
    \end{pmatrix}$, $f_2= \begin{pmatrix}
        0\\1\\2
    \end{pmatrix}$ in $\mathbb{R}^3$. Let $v = \begin{pmatrix}1 \\ 1 \end{pmatrix}$ in the f-basis. Then, the coordinates of
    v in $\mathbb{R}^3$ is $v= 1 \cdot f_1 + 1 \cdot f_2 = \begin{pmatrix} 1\\3\\5 \end{pmatrix}$
\end{example}
\begin{example}(Change of basis matrix)
    Let $\begin{pmatrix}
        1 & 0 \\ 2&1 \\ 3&2 
    \end{pmatrix}$ be a matrix which changes f-coord to e-coord. 
    $$\begin{pmatrix}
        1 & 0 \\ 2&1 \\ 3&2 
    \end{pmatrix} \begin{pmatrix}
        1\\1
    \end{pmatrix} = \begin{pmatrix}
        1\\3\\5 
    \end{pmatrix}$$

\end{example}
\begin{definition}(\textbf{Change of basis matrix})
    Let $e_1, ... e_n $ be a basis of V. Let  $f_1, ... f_n $ be vectors in V written in e-coordinates. Then, $Q=( f_1, ... f_n )$ is 
    called the change of basis matrix $f-coord\rightarrow e-coord$. Where $$Q\begin{pmatrix}
        1 \\ 0\\ \vdots \\ 0 
    \end{pmatrix} = f_1$$
\end{definition}
\begin{remark}
    Both $e= e_1, ... e_n $ and $f=  e_1, ... f_n $ are bases of V.
    $$Q = (f_1,..., f_n)\; :\; f-coord \rightarrow e- coord$$
    $$P = (f_1,..., f_n)\; :\; e-coord \rightarrow f- coord$$ 
    $$Q= P^{-1}$$
\end{remark}
\begin{corollary}
    Let $f_1 ... f_n$ be a basis of V, written in e-coord. Then, $(Q=f_1 ... f_n)$ is invertible. That is, if the vectors 
    form a basis, the matrix is invertible.   
\end{corollary}
\begin{theorem}
    Let $T : V\rightarrow V $ a linear transformation, $e_i$ a basis of V, $f_i$ a basis of V, both of finite dimension.
    Let $A=$(matrix of T in e-basis),$B=$(matrix of T in f-basis), 
    \newline $Q=$ (change of basis matrix f-coord $\rightarrow$ e-coord). 
    Then, $$B = Q^{-1}A Q$$
\end{theorem}
\section{Application: Coding Theory}
Let p a prime number, $\mathbb{F}_p =$ Field with p elements. Let k an integer, and the elements of $\mathbb{F}^k_p$, a vector space, are called words
of length k.  
\begin{example}
    $p=2$, $0101$ a word of length 4 in $\mathbb{F}^4_2$. 
    
    Goal: transmit words through a channel. $$(word)\rightarrow_{\substack{encoding}} (longer \; word)\rightarrow_{\substack{transmit}}(modified \; word)\rightarrow_{\substack{decode}}(original \; word)$$

\end{example}

\begin{definition}(Linear codes)
    A linear encoding is an injective map 
    $$\phi \; :\;\mathbb{F}^k_p \rightarrow \mathbb{F}^n_p$$
    with $im(\phi) =$ code words
\end{definition}

\begin{example}
    Let $\phi = \frac{Id}{Id}$
    $$\phi :\mathbb{F}^k_p \rightarrow \mathbb{F}^{2k}_p$$
    e.g. with $p=2$, $\phi(0101) = 01010101$
    $$0101 \rightarrow_E 01010101 \rightarrow_T 01010100$$
\end{example}
\subsection{Problems}
\begin{enumerate}
    \item Make the code words as small as possible.
    \item Make sure to recover the original words from the transmitted words, knowing the transmition error rate. 
\end{enumerate}
\begin{definition}(parity code)
    Let $p=2$
    $$0101 \rightarrow 01010 $$
    The last 0 in the image is a parity ($\sum$ of digits modulo 2)
    
\end{definition}
\begin{example}
    Let $\phi =\mathbb{F}^k_2\rightarrow \mathbb{F}^{k+1}_2$. 
    $$\phi = \begin{pmatrix}
        1&0&0 &\hdots &0 \\0&1&0 &\hdots &0 \\ \vdots \\ 0&0&0 &\hdots &1 \\ 1&1&1&\hdots &1
    \end{pmatrix}$$
    (adds parity bit to word)
\end{example}
\subsection{Hamming Code}
Acts on words of length 4, makes them into words of length 7.
$$\phi = \begin{pmatrix} 
    1&0&0&0 \\ 0&1&0&0 \\0&0&1&0 \\ 0&0&0&1\\ 0&1&1&1 \\ 1&0&1&1 \\ 1&1&0&1 
\end{pmatrix}$$
$$\phi(0101) = 0101010$$
\begin{definition}(Hamming distance on $\mathbb{F}^n_p$)
    Let $u, v \in\mathbb{F}^n_p$, $d(u,v)=\# \{i : u_i \neq v_i\}$. e.g. 
    $$d(0000,1111) = 4$$
    $$d(0000,0001) = 1$$ 
\end{definition}
\begin{example}
    Show the triangle inequality for the hamming distance. 
\end{example}
\begin{definition}(Hamming norm)
    $||u|| = d(u,0)$
    
\end{definition}
\begin{example}
    Let $\phi = parity \; bit$, then $im(\phi) = \{v \in \mathbb{F}^{n+1}_2 \; : \; \sum^{n+1}_1 v_i= 0\}$
    \begin{proof}
        Let $v\in im(\phi)$, then $v_{n+1} = \sum^n_1 v_i$, so 
        $$\sum^{n+1}_1 v_i = 2\left(\sum^n_1 v_i\right) = 0$$
        $$im(\phi) \subseteq \left\{ v : \sum^{n+1}_1 v_i=0\right\}$$
        Other inclusion is left as an exercise. 
    \end{proof}
\end{example}
\begin{definition}(Separation between code words)
    $Sep(\phi) = \min \{ d(u,v) : u,v \in \; im(\phi), \; u\neq v\} $, with d the Hamming distance. 
\end{definition}
\begin{example}
    Show $Sep(\phi) = \min \left\{ ||u|| \; :\; u \in im(\phi), u\neq 0\right\}$
\end{example}
\section{Revisiting linear systems (chpt 3 in textbook)}
End of material for midterm. 
\subsection{Hyperplanes}
\begin{definition}
    Let V a vector space (f.d.). A hyperplane is a subspace of $dim(V)-1$. Equivalently, a hyperplane V in a vector 
    space W is any subspace such that W/V is one-dimensional. Equivalently, a hyperplane is the linear 
    transformation kernel of any nonzero linear map from the vector space to the underlying field.
\end{definition}
\begin{example}
    $x+y+z = 0$ a hyperplane in $\mathbb{R}^3$. 
    \newline
    \textbf{Implicit definition}
    $$H = \left\{ \begin{pmatrix} x\\y\\z \end{pmatrix} \;\middle| \; x+y+z = 0 \right\}$$
    \textbf{Parametric definition}
    \newline 
    $\begin{pmatrix} x\\y\\z \end{pmatrix} \in H$, then 
   
    $$\begin{pmatrix} x\\y\\z \end{pmatrix} = \begin{pmatrix} -y-z\\y\\z \end{pmatrix} = y\begin{pmatrix} -1\\1\\0 \end{pmatrix}+z \begin{pmatrix} -1\\0\\1 \end{pmatrix}$$
    $$H = span \left\{ \begin{pmatrix} -1\\1\\0 \end{pmatrix}, \; \begin{pmatrix} -1\\0\\1 \end{pmatrix} \right\}$$
\end{example}
\begin{definition}(System of k equations)
    $$\left\{\begin{matrix} a_{11}x_1 + \hdots + a_{1n}x_n = 0 \\ \vdots \\a_{k1}x_1 + \hdots + a_{kn}x_n = 0 \end{matrix} \right\}$$
\end{definition}
\begin{remark}
    The intersection of subspaces is a subspace.
\end{remark}

\begin{example}
    To find the solution of a system: Gaussian elimination. 

    Elementary operations:
    \begin{enumerate}
        \item swap two rows $R_i \leftrightarrow R_j$
        \item Mutltiply by a non zero constant $R_i \rightarrow \lambda R_i$ 
        \item Add to a row a multiple of a different row $R_i \rightarrow R_i + \lambda R_j$
    \end{enumerate}
\subsection{Elementary Matrices}
\begin{definition}
    An elementary matrix is a matrix which differs from the identity matrix by one single elementary row operation. 
    The elementary matrices generate the general linear group $GL_n(F)$ when $F $ is a field. Left multiplication
     (pre-multiplication) by an elementary matrix represents elementary row operations, while right multiplication 
     (post-multiplication) represents elementary column operations.
\end{definition}
\begin{enumerate}
    \item Swap $R_i \leftrightarrow R_j$
    \begin{remark}
        To act on the rows, multiply by an elementary matrix on the left, to act on the columns, multiply by a elementary matrix on the right.
        \begin{example}
            $$\begin{pmatrix}
                1&0&0\\
                0&0&1\\
                0&1&0
            \end{pmatrix} 
            \begin{pmatrix}
                1&2&3\\4&5&6\\7&8&9
            \end{pmatrix} = \begin{pmatrix}
                1&2&3\\7&8&9\\4&5&6
            \end{pmatrix}$$
        \end{example}
    \end{remark}
    \item $R_i \rightarrow \alpha R_i$, $\alpha \neq 0$
    $$\begin{pmatrix}
        1&0&0 \\ 0&2&0 \\ 0&0&1 
    \end{pmatrix}
    \begin{pmatrix}
        1&2&3 \\ 1&2&3\\1&2&3 
    \end{pmatrix} = 
    \begin{pmatrix}
        1&2&3 \\2&4&8\\1&2&3
    \end{pmatrix}$$
    \item $R_i \rightarrow R_i + \alpha R_j$, $i \neq j$
\end{enumerate}
\end{example}
    \begin{theorem}
        Elementary matrices are invertible, and the inverse of an
elementary matrix is an elementary matrix of the same type.
    \end{theorem}
\begin{definition}(REF)
    Let A be a matrix, then $\exists (E_1, \hdots , E_n)$ elementary matrices such that 
    $$E_n \circ \hdots \circ E_2\circ E_1 \circ A = REF(A)$$
\end{definition}

\begin{theorem}
    Let $A \in M_{m\times n}(F)$, and suppose that B is obtained from A by performing an elementary 
    row [column] operation. Then there exists an $m \times m$ $[n \times n]$ elementary matrix E such that 
    $B =EA$ $[B = AE]$. In fact, E is obtained from $I_m$ $[I_n]$ by performing the same elementary row [column]
     operation as that which was performed on A to obtain B. Conversely, if E is an elementary $m \times m$ $[n \times n]$
      matrix, then $EA$ $[AE]$ is the matrix obtained from A by performing the same elementary row [column] operation 
      as that which produces E from $I_m$ $[I_n]$.
\end{theorem}

\subsection{Rank theorem for systems}
Let $V = K^n$, 
$$X = \left\{\begin{matrix}
    eq_1 = 0 \\ \vdots \\ eq_k = 0 
\end{matrix} \right\} \iff AX = 0$$
We have k equations, n variables. Let $S = $(subspace of solutions). The dimension of S can be found with the theorem below. 
We view A as a linear transformation matrix:
$$A : K^n \rightarrow K^k$$
A is a matrix with k rows and n columns. 
$$S = ker(A) = \{x \in K^n :\; AX =0\}$$


\begin{theorem}
    $$dim(S)= (num\; of \;variables\; (n)) - ("True" \;num \;of \;equations),$$ where S is the set of solutions. 
    To find the true number of equations, start with A, apply gaussian elimination, and the non zero rows in the REF is 
    the number of true equations.
\end{theorem}

\begin{definition}(Row space of A)
    Take only the rows of the matrix A.
    $$Row(A)  = span\{r_1, ..., r_k \} \subseteq K^n,$$ where 
    $$A = \begin{pmatrix}
        r_1 \\ \vdots \\ r_k 
    \end{pmatrix}$$
\end{definition}
\begin{definition}(Column space of A)
    Take only the columns of A. 
    $$col(A) = span\{ c_1 , ...,c_k\}$$
\end{definition}

\begin{remark}
    True number of equations: $dim(Row(A))$
\end{remark}
\begin{theorem}(old rank theorem)
    $$dim(V) = dim(ker(A)) + dim(Col(A))$$
\end{theorem}

\begin{definition}
    If $A\in M_{n\times m}(K)$, we define the rank of A to be the rank 
    of the linear transformation $L_A: K^n \rightarrow K^m$.
\end{definition}
\begin{theorem}
    Elementary row and column operations are rank preserving.
\end{theorem}

\begin{theorem}
    The rank of any matrix equals the maximum number of its linearly independent columns; that is, the rank of a 
    matrix is the dimension of the subspace generated by its columns.
    $$Rank(A) = dim(Im(A)), \text{ where } Im(A) = Col(A)$$
\end{theorem}

\begin{theorem}(Rank transpose theorem)
    $$rank(A) = rank(A^t)$$
    $$Col(A) = Row(A^t)$$

    \begin{proof}
        Will come later.
    \end{proof}
\end{theorem}
\begin{corollary}
    $dim(Row(A) = dim(Col(A)))$
\end{corollary}
\begin{corollary}
    Let A be an $n\times m$ matrix of rank r. Then, there exist invertible matrices $B_{m\times m}$ and $C_{n \times n}$ such that
    $D = BAC$
\end{corollary}
\begin{theorem}
    The rank of any matrix equals the maximum number of its linearly independent columns; that is, 
    the rank of a matrix is the dimension of the subspace generated by its columns. Moreover, the rows and columns of any matrix generate subspaces of the same
    dimension, numerically equal to the rank of the matrix.
\end{theorem}
\begin{theorem}
    Let $A$ be an $m \times n$ matrix, if $P$ and $Q$ are invertible $m \times m$ and $n\times n$ matrices, respectively, then

    (a) $rank(AQ) = rank(A),$

    (b) $rank(PA) = rank(A)$, and therefore,

    (c) $rank(PAQ) = rank(A)$.
\end{theorem}
\begin{theorem}(Reduction to a Projection)
    Let A be an ($m \times n$) matrix of rank r. Then $r \leq m$, $r \leq n$, and $\exists E_1, \hdots , E_n$ and $F_1, \hdots , F_m$ elementary matrices such that 
    $$E_n \circ \hdots \circ E_1 \circ A \circ F_1 \circ \hdots \circ F_m = \begin{pmatrix}
        Id_r & 0_1\\0_2&0_3
    \end{pmatrix} = D$$
    where $0_1, 0_2,0_3$ are zero matrices. Thus, $D_{ii} =1 $ for $i\leq r$ and $D_{ij} = 0$ otherwise. 
    That is, the elementary matrices on the left act on the columns and the elementary matrices on the right act
    on the rows to reduce A to a projection matrix. 
\end{theorem}
\begin{corollary}
    Let A be an m x n matrix of rank r. Then there exist
invertible matrices B and C of sizes $m\times m$ and $n \times n$, respectively, such that  $D = BAC$, where
$$D =\begin{pmatrix}
    Id_r & O_1 \\O_2 &O_3 
\end{pmatrix} $$
is the $m\times n$ matrix in which $O_i$ are zero matrices.
\end{corollary}
\begin{theorem}
    $rank(A) = rank(A^t)$
\end{theorem}
\begin{proposition}
    $rank(A) = r $ if $ A \rightarrow \begin{pmatrix}
        Id_r & 0\\ 0&0
    \end{pmatrix}$
\end{proposition}
\begin{example}
    Consider the matrix $$A = \begin{pmatrix}
        0&2&4&2&2\\4&4&4&8&0\\8&2&0&10&2\\6&3&2&9&1
    \end{pmatrix} \xrightarrow{\text{elementary operations}} 
    \begin{pmatrix}
        1&0&0&0&0\\0&1&0&0&0\\0&0&1&0&0\\0&0&0&0&0
    \end{pmatrix} = D$$
    So, since we know $rk(A) = rk(D)$, we get that $rank(A)= 3$
\end{example}

\begin{lemma}
    If E, F are invertible, then $rank(A) = rank(EAF)$, which implies the proposition above by induction on the theorem. 
\end{lemma}
\begin{corollary}
    $rk(A) =rk(A^t)$

    \begin{proof}
        
    
    $$rk(A) = E_n \circ \hdots \circ E_1 \circ A \circ F_1 \circ \hdots \circ F_m = \begin{pmatrix}
        Id_r & 0\\ 0&0
    \end{pmatrix} = r $$
    $$ rk(A^t) = (E_n \circ \hdots \circ E_1 \circ A \circ F_1 \circ \hdots \circ F_m)^t = \begin{pmatrix}
        Id_r & 0\\ 0&0
    \end{pmatrix}^t = r $$
    \end{proof}
\end{corollary}
\subsection{Invertible matrices}
\begin{theorem}
    A is an $n \times n$ invertible matrix $\iff rank(A) = n$
\end{theorem}
\begin{theorem}
    Every invertible matrix is the product of elementary matrices. 
\end{theorem}
\begin{example}
    To find the inverse of a matrix; let A be an $n\times n$ invertible matrix, and consider the $n \times 2n $ augmented matrix $C = (A | I_n)$. Then, 
    $$A^{-1} C = (A^{-1} A | A^{-1} I_n) =(I_n | A^{-1})$$
    That is, it is possible to transform the augmented matrix $C = (A | I_n)$ into $(I_n | A^{-1})$ with a finite number of elementary matrices, finding the inverse matrix. 
\end{example}
\begin{theorem}
    Let $Ax = b$ be a system of n linear equations in n unknowns. If A is invertible, then the system has exactly 
    one solution, namely,$A^{-1} b$. Conversely, if the system has exactly one solution, then A is invertible.
\end{theorem}
\begin{definition}
    Let V and W be vector spaces. We say that V is isomorphic to W if there exists a linear transformation 
    $T: V \rightarrow W$ that is invertible. Such a linear transformation is called an isomorphism from V onto W.
\end{definition}
\begin{theorem}
    Let V and W be finite-dimensional vector spaces (over the same field). Then V is isomorphic to W if and only if $dim(V) = dim(W)$. 
\end{theorem}
\begin{theorem}
    Let $Ax = b$ be a system of linear equations. Then the system is consistent if and only if $rank(A) = rank(A|b)$.
\end{theorem}
\subsection{The General Linear Group}
\begin{theorem}
    Let $A_{n\times n}$ an invertible matrix. Then, $\exists E_1, \hdots, E_k $ such that 
    $$A = E_1\hdots E_k$$
    viz. any invertible matrix is a product of elementary matrices. 
    \begin{proof}
        $A\rightarrow E_n \hdots E_1 A F_1 \hdots F_m = Id_r \implies rk(A) = r$
    \end{proof}
\end{theorem}
\begin{definition}
    $GL_n(K)$ is the set of all invertible matrices ($n \times n$), this is a group.
\end{definition}

\begin{corollary}
    $GL_n(K)$ is generated by the elementary matrices. It is a symmetric generating set $S$ because $S =S^{-1}$, as 
    inverses of elementary matrices are also elementary matrices.
\end{corollary}
\begin{definition}(Generating set)
    Let G a group, then S is a generating set if every $s\in G$ can be written as $s = \pi s_i$, where $s_i \in S\cup S^{-1}$. 
    A generating set is a set of vectors that spans a vector space. All the vectors in the space can be written as a linear 
    combination of the vectors of the generating set. 
\end{definition}
\subsection{The symmetric group}
\begin{definition}
    Let X a set. 
    $Sym(X) =$ the set of all bijections $X\rightarrow X$. $X = \{ 1,..., n\}$, $S_n = Sym(X)$
\end{definition}
\begin{definition}(Cycle decomposition of a permutation)
    Let $\sigma \in S_n \implies \sigma$ is a bijection. Any bijection can be written as a product of disjoint cycles in a unique way. 
    e.g. $$\sigma = (x_1 x_2x_3x_4)(x_5x_6x_7)$$
    $$\sigma = (1\;2\;3\;4)(5\;6\;7) = (5\;6\;7) (1\;2\;3\;4)$$
\end{definition}
\begin{lemma}
    Every cycle is a product of transpositions which is a transformation which permutes two elements, while 
    fixing every other element, e.g. $(xy)$.
\end{lemma}
\begin{example}
    $(123) = (13)(12) = (32)(31)$
\end{example}
\begin{theorem}
    $S_n$ is generated by transpositions.
\end{theorem}
\subsection{Permutation matrices}
$$S_n \leftrightarrow GL_n(K)$$
Let $\sigma \in S_n$. Permutation matrix:
$$A_\sigma = \left\{ \begin{matrix}
    a_{ij}=1 & if & \sigma(j) = i \\ a_{ij}= 0 & if& \sigma(j) \neq i
\end{matrix}\right.$$

\begin{example}
    For example, the permutation matrix $P_{\pi}$ corresponding to the permutation 
    $$\pi=\begin{pmatrix} 1 & 2 & 3 & 4 & 5 \\ 1 & 4 & 2 & 5 & 3 \end{pmatrix}$$
    is
    $$ P_{\pi} = 
    \begin{bmatrix}
    \mathbf{e}_{\pi(1)} \\
    \mathbf{e}_{\pi(2)} \\
    \mathbf{e}_{\pi(3)} \\
    \mathbf{e}_{\pi(4)} \\
    \mathbf{e}_{\pi(5)} 
    \end{bmatrix}
    =
    \begin{bmatrix}
    \mathbf{e}_{1} \\
    \mathbf{e}_{4} \\
    \mathbf{e}_{2} \\
    \mathbf{e}_{5} \\
    \mathbf{e}_{3} 
    \end{bmatrix}
    =
    \begin{bmatrix} 
    1 & 0 & 0 & 0 & 0 \\
    0 & 0 & 0 & 1 & 0 \\
    0 & 1 & 0 & 0 & 0 \\
    0 & 0 & 0 & 0 & 1 \\
    0 & 0 & 1 & 0 & 0 
    \end{bmatrix}.
    $$
    Observe that the $j^{th}$ column of the $I_5$ identity matrix now appears as the $\pi(j)$th column of $P_{\pi}$.
     

\end{example}
\begin{proposition}
    The map 
    $$S_n \rightarrow GL_n$$
    $$\sigma \rightarrow A_\sigma$$
    is an injective homomorphism. 
\end{proposition}
\section{The Determinant}
\subsection{Determinant of order 2}
\begin{definition}
    If $$A = \begin{pmatrix}
        a&b\\c&d 
    \end{pmatrix}$$
    is a $2 \times 2$ matrix with entries from a field $F$, then we define the determinant of $A$, denoted $\det(A)$ or $|A|$, to be the scalar $ad - bc$.
\end{definition}
The determinant in itself is not a linear transformation. Nevertheless, the determinant does possess 
an important linearity property, which is explained in the following theorem.
\begin{theorem}
    The function $\det: M_{2\times 2} (F) \rightarrow F$ is a linear function of each row of a $2 \times 2$ 
    matrix when the other row is held fixed. That is, if u, v, and w are in $F^2$ and $k$ is a scalar, then
    $$\det \begin{pmatrix}
        u+kv \\ w 
    \end{pmatrix} = \det \begin{pmatrix}
        u \\w 
    \end{pmatrix} + k \det \begin{pmatrix}
        v \\w 
    \end{pmatrix}$$
\end{theorem}
\subsection{Determinant of order n}
\begin{definition}
    Given $A \in M_{n\times n} (F)$, for $n > 2$, denote the $(n - 1) \times (n - 1)$ matrix obtained from $A $ 
    by deleting row i and column j by $\tilde{A}_{ij}$.
\end{definition}
\begin{definition}(\textbf{Laplace Cofactor Expansion})
    Let $A \in M_{n\times n}(F)$. If $n = 1$ so that $A = (A_{11})$, we define $\det(A) = (A_{11})$. For $n > 2$, 
    we define $\det(A)$ recursively as
    $$\det(A) =\sum^n_{j = 1} (-1)^{1+j} A_{1j} \cdot \det(\tilde{A}_{1j})$$
    The scalar $(-1)^{1+j} \cdot \det(\tilde{A}_{1j})$ is called the \textbf{cofactor} of the entry of 
    $A$ in row $i$, column $j$. 

    Letting 
    $$c_{ij} = (-1)^{1+j}  \cdot \det(\tilde{A}_{1j}),$$ 
    we express the formula for the determinant of $A$ as 
    $$\det(A) = A_{11} c_{11} + A_{12}c_{12} + ... + A_{1n} c_{1n}.$$
    Thus the determinant of A equals the sum of the products of each entry in row 1 of A multiplied by its cofactor.
\end{definition}
\begin{theorem}
    The determinant of an $n \times n$ matrix is a linear function of each row when the remaining rows are held fixed. 
    That is, for $1 < r < n$, we have
    $$\det \begin{pmatrix}
        a_1 \\ \vdots \\ a_{r-1} \\ u+kv \\ a_{r+1} \\ \vdots \\ a_n 
    \end{pmatrix} = 
    \det \begin{pmatrix}
        a_1 \\ \vdots \\ a_{r-1} \\ u \\ a_{r+1} \\ \vdots \\ a_n 
    \end{pmatrix} 
    + k \det  \begin{pmatrix}a_1 \\ \vdots \\ a_{r-1} \\ v \\ a_{r+1} \\ \vdots \\ a_n 
    \end{pmatrix}$$
\end{theorem}
\begin{corollary}
    If $A \in M_{n \times n} (F)$ has a row consisting entirely of zeros, then $det(A) = 0$.
\end{corollary}
\begin{theorem}
    The determinant of a square matrix can be evaluated by cofactor expansion along any row.
\end{theorem}


\begin{theorem} 
    The following rules summarize the effect of an elementary row operation on the determinant of a matrix $A \in M_{n \times n} (F)$. 
    \begin{enumerate}
        \item Let $A \in M_{n\times n}(F)$, and let $B$ be a matrix obtained by adding a multiple of one row of $A$
        to another row of $A$. Then $\det(B) = \det(A)$.
        \item If $A \in M_{n\times n}(F)$ and $B$ is a matrix obtained from $A$ by
        interchanging any two rows of $A$, then $\det(-B) = - \det(A)$. 
        \item If B is a matrix obtained by multiplying a row of A by a nonzero scalar k, then $\det(B) = k \det(A)$.
    \end{enumerate}
    These facts can be used to simplify the evaluation of a determinant along with the following theorem.
\end{theorem}
\begin{theorem}
    The determinant of an upper triangular matrix is the product of its diagonal entries.
\end{theorem}
\begin{corollary}
    If $A \in M_{n\times n}(F)$ has two identical rows, then $det(A) = 0$.
\end{corollary}
\begin{corollary}
    If $A \in M_{n \times n} (F)$ has rank less than n, then $det(A) = 0$.
\end{corollary}

\begin{remark}
    Since we can transform any square matrix into an upper triangular using elementary row operations, to evaluate its determinant:
    
    (1) reduce to upper triangular 

    (2) multiply the diagonal entries together

    (3) multiply the obtained scalar by -1 if you interchanged rows

    (4) the result is the determinant.
\end{remark}

\subsection{Properties of the determinant}
\begin{theorem}
    For any $A,B \in M_{n \times n}(F)$, $det(AB) = det(A)\cdot det(B)$.
\end{theorem}
\begin{corollary}
    A matrix $A \in M_{n\times n} (F)$ is invertible if and only if $\det (A) \neq 0 $. Furthermore, 
    if A is invertible, then $\det(A^{-1}) = \frac{1}{\det(A)}$.
\end{corollary}

\begin{theorem}
    For any $A \in M_{n\times n} (F )$, $\det(A^t) = \det(A) $.
\end{theorem}

\subsection{Lebnitz formula}
\begin{theorem}
    There exists a unique homomorphism
    $$\varepsilon: S_n \rightarrow \mathbb{Z}/2\mathbb{Z}$$
    and it is surjective. It verifies
    $$\varepsilon (ab) = -1$$
    $$\varepsilon (abc) = 1$$
    $$\varepsilon (abcd) =-1$$
    And, 
    \begin{align*}
        (123)&=(32)(31) \\
        \varepsilon(123)&= \varepsilon(32) \varepsilon(31) \\
        (1)&=(-1)(-1)
    \end{align*}
\end{theorem}
\begin{definition}
    The kernel of the $\varepsilon$ homomorphism is the alternating group, viz. the permutations which can be expanded into 
    an even number of transpositions get sent to 0. 
    $$A_n = \ker(\varepsilon)$$
\end{definition}
Let $A\in M_n(k)$, then 
$$\det(A) = \sum_{\sigma \in S_n} \underbrace{\varepsilon(\sigma) a_{\sigma(1)1} \hdots a_{\sigma(n)n}}_{a_{21} a_{32} a_{13} \text{ if $\sigma = (123)$}}$$
\begin{example}
    Let $A_\sigma = $ permutation matrix. 
    \begin{align*}\det(A_\sigma) &= \sum_{z \in S_n} \varepsilon(z) \underbrace{a_{z(1)1}\hdots a_{z(n)n}}_{a_{ij} = 1 \iff \sigma(j) = 1} \\ 
            &= \varepsilon(\sigma ) a_{\sigma(1)1} \hdots a_{\sigma(n)n}\\
            &= \varepsilon (\sigma)
    \end{align*}
\end{example}
\begin{example}
    let $A_\sigma$ a permutation matrix. 
    $$e_i = \begin{pmatrix}
        0\\1\\0\\\vdots \\0
    \end{pmatrix}$$

    $$A_{\sigma(e_i)}= e_{\sigma(i)}$$
    If $\sigma = (123),$
    $$\begin{pmatrix}
        0&0&1\\1&0&0\\0&1&0
    \end{pmatrix} = A_\sigma $$
    $$\det(A_\sigma) = \varepsilon(\sigma)$$
\end{example}
\begin{example}
    $$\det \begin{pmatrix}
        0&0&\lambda_1 \\ \lambda_2 &0&0\\ 0&\lambda_3&0
    \end{pmatrix} = \varepsilon(123) \lambda_1 \lambda_2 \lambda_3 =  \lambda_1 \lambda_2 \lambda_3$$
\end{example}
\begin{example}
    $$\det \begin{pmatrix}
        1 &\hdots &a \\
        \vdots&\ddots&\vdots\\
        0&\hdots&1 
    \end{pmatrix}$$
    It is an upper triangular matrix, hence a permutation $\sigma$ contributes to the sum if and only 
    if $\sigma(i) \leq i$, because $a_{\sigma(n)n} = 0 $ if $\sigma(n) >n$.
\end{example}
\begin{lemma}
    Suppose $\sigma(i)\leq i \; \forall i $, i.e. upper triangular matrix. Then $\sigma = id$.
    \begin{proof}
        Let $\sigma$ such that $\sigma(i) \leq i$ 
        \begin{align*}
            \sigma(1) \leq 1 &\rightarrow \sigma(1) = 1 \\
            \sigma(2) \leq 2 &\rightarrow \sigma(2) = 2 \\ 
            \sigma(3) \leq 3 &\rightarrow \sigma(3) = 3 \\
            \vdots \\
            \sigma(n) = id &
        \end{align*}
    \end{proof}
\end{lemma}
\begin{definition}
    A definition of the determinant:

    $$\det(A) = \sum_{\sigma\in S_n}\text{sgn}(\sigma)\prod\limits_{i=1}^n a_{i,\sigma_i}$$
In hopefully simpler terms, we range over all possible "patterns" (*ways of picking $n$ entries such that every row 
and every column is represented exactly once*), computing the product of the entries in the pattern, applying a sign 
change depending on the number of pairs of entries in the pattern where one is above and to the right of other entries 
in the pattern (*if odd, change sign.  if even, keep sign the same*), and then summing.
\end{definition}

\section{Diagonalization}

\subsection{Eigenvalues and Eigenvectors}

\subsection{Multilinearity}
\begin{definition}
    A \textbf{multilinear map} is a function of several variables that is linear separately in each variable.
    More precisely, a multilinear map is a function
    $$f : V_1 \times \hdots \times V_n \rightarrow W,$$
    where $ V_1 \times \hdots \times V_n$ and W are vector spaces, with the following property: for each 
    $i$, if all of the variables but $v_i$ are held constant, then $f(v_1,...,v_i,...v_n)$ is a linear function of $v_i$.
    A multilinear map of one variable is a linear map, and of two variables is a bilinear map. More generally, 
    a multilinear map of k variables is called a k-linear map. If the codomain of a multilinear map is the field 
    of scalars, it is called a multilinear form. 
\end{definition}
\begin{definition}
Let $V, W $ and $X$ be three vector spaces over the same base Field $F$. A bilinear map is a function
$$B : V \times W \to X$$
such that for all $w \in W$, the map $B_w$
$$v \mapsto B(v, w)$$
is a linear map from $V$ to $X,$ and for all $v \in V$, the map $B_v$
$$w \mapsto B(v, w)$$
is a linear map from $W$ to $X.$ In other words, when we hold the first entry of the bilinear map fixed while letting the second entry vary, the result is a linear operator, and similarly for when we hold the second entry fixed.

Such a map $B$ satisfies the following properties.
\begin{enumerate}
    \item For any $\lambda \in F$, $B(\lambda v,w) = B(v, \lambda w) = \lambda B(v, w).$
    \item The map $B$ is additive in both components: if $v_1, v_2 \in V$ and $w_1, w_2 \in W,$ then $B(v_1 + v_2, w) = B(v_1, w) + B(v_2, w)$ and $B(v, w_1 + w_2) = B(v, w_1) + B(v, w_2).$
\end{enumerate}
    
\end{definition}
\end{document}


